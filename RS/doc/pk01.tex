\documentclass[a4paper,10pt]{scrartcl}
\usepackage[utf8]{inputenc}
\usepackage[ngerman]{babel}
\usepackage[T1]{fontenc}
\usepackage{amsmath}
\usepackage{amssymb}
\usepackage{graphicx}
\usepackage{qtree}

\begin{document}
\title{Praktikum Rechnerstrukturen 01}
\author{Jan Lukas Deichmann / Jan-Tjorve Sobieski}
\graphicspath{{images/}}
\maketitle
\noindent
\\\underline{1.2c i}\\\\
Gesucht: $X_{3} \land X_{2} \land X_{1} \land X_{0}$ (4AND)\\\\
$(X_{3} \land X_{2}) \land (X_{1} \land X_{0})$ (Assoziativität) \\
$\Leftrightarrow$ $X_{3} \land X_{2} \land (X_{1} \land X_{0})$ (Assoziativität)\\
$\Leftrightarrow$ $X_{3} \land X_{2} \land X_{1} \land X_{0}$\\\\
Gesucht: $X_{2} \land X_{1} \land X_{0}$ (3AND)\\\\
$(X_{2} \land X_{1}) \land X_{0}$ (Assoziativität) \\
$\Leftrightarrow$ $X_{2} \land X_{1} \land X_{0}$ \\\\
\underline{1.2c ii}
\begin{table}[h]
\begin{tabular}{l r}
    \Tree[.$\lor$ [.$X_{0}$ ]
    [.$\lor$ [.$X_{2}$ ]
        [.$X_{1}$ ]]]
 & \Tree[.$\lor$3 [.$X_{0}$ ]
 [.$X_{1}$ ] [.$X_{2}$ ] ]
\end{tabular}
\end{table}
\\Die Tiefe des Ausdrucks verändert sich nicht, da ein normaler Operatorbaum mit einem erweiterten Operatorbaum nicht verglichen werden kann.
\end{document}
