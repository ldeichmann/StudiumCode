\documentclass[a4paper,10pt]{scrartcl}
\usepackage[utf8]{inputenc}
\usepackage[ngerman]{babel}
\usepackage[T1]{fontenc}
\usepackage{amsmath}
\usepackage{amssymb}
\usepackage{graphicx}
\usepackage{qtree}

\begin{document}
\title{Praktikum Rechnerstrukturen 01}
\author{Jan Lukas Deichmann / Jan-Tjorve Sobieski}
\graphicspath{{images/}}
\maketitle
\noindent
\\\underline{1.2c i}\\\\
Gesucht: $X_{3} \land X_{2} \land X_{1} \land X_{0}$ (4AND)\\\\
$(X_{3} \land X_{2}) \land (X_{1} \land X_{0})$ (Assoziativität) \\
$\Leftrightarrow$ $X_{3} \land X_{2} \land (X_{1} \land X_{0})$ (Assoziativität)\\
$\Leftrightarrow$ $X_{3} \land X_{2} \land X_{1} \land X_{0}$\\\\
Gesucht: $X_{2} \land X_{1} \land X_{0}$ (3AND)\\\\
$(X_{2} \land X_{1}) \land X_{0}$ (Assoziativität) \\
$\Leftrightarrow$ $X_{2} \land X_{1} \land X_{0}$ \\\\
\underline{1.2c ii}
\begin{table}[h]
\begin{tabular}{l r}
    \Tree[.$\lor$ [.$X_{0}$ ]
    [.$\lor$ [.$X_{2}$ ]
        [.$X_{1}$ ]]]
 & \Tree[.$\lor$3 [.$X_{0}$ ]
 [.$X_{1}$ ] [.$X_{2}$ ] ]
\end{tabular}
\end{table}
\\Die Tiefe des Ausdrucks verändert sich nicht, da ein normaler Operatorbaum mit einem erweiterten Operatorbaum nicht verglichen werden kann.
\newpage
\noindent
\underline{1.2d}\\\\
Sei $\mathbb{B}$ = \{0,1\} und D $\subseteq$ $\mathbb{B}^n$\\\\
Sei $\mathbb{B}_{4,1}$:= \{ f; f: $\mathbb{B}^4$ $\rightarrow$ $\mathbb{B}^1$\}\\\\
$f(X_{3},X_{2},X_{1},X_{0})$ = \\\( (\lnot X_{2} \land X_{1} \land X_{0}) \lor (\lnot X_{3} \land X_{1} \land X_{0}) \lor (\lnot X_{3} \land X_{2} \land X_{0}) \lor (X_{2} \land \lnot X_{1} \land X_{0})\)\\\\\\
\underline{1.2e}\\\\
Sei $\mathbb{B}$ = \{0,1\} und D $\subseteq$ $\mathbb{B}^n$\\\\
Sei $\mathbb{B}_{4,1}$:= \{ f; f: $\mathbb{B}^4$ $\rightarrow$ $\mathbb{B}^1$\}\\\\
$f(X_{3},X_{2},X_{1},X_{0})$ = \\\( (\lnot X_{2} \land X_{1} \land X_{0}) \lor (\lnot X_{3} \land X_{1} \land X_{0}) \lor (\lnot X_{3} \land X_{2} \land X_{0}) \lor (X_{2} \land \lnot X_{1} \land X_{0}) \\\lor (\lnot X_{3} \land \lnot X_{2} \land X_{1} \land \lnot X_{0})\)\\\\
\newpage
\noindent
\underline{1.3}\\\\
Beschreibung der Funktion:\\
Ein Volladdierer, aufgebaut aus zwei Halbaddierern.\\\\
\underline{1.4a i}
\begin{tabbing}
\underline{$X_{3}$ $X_{2}$ $X_{1}$ $X_{0}$ Y} \\
$X_{3}$ \=$X_{2}$ \=$X_{1}$ \=$X_{0}$ \=Y \kill
0 \>0 \>1 \>1 \>1\\
0 \>1 \>0 \>1 \>1\\
0 \>1 \>1 \>0 \>1\\
1 \>0 \>0 \>1 \>1\\
1 \>0 \>1 \>0 \>1\\
1 \>1 \>0 \>0 \>1\\\\\\
\end{tabbing}
\underline{1.4a ii}\\\\
 $f(X_{3},X_{2},X_{1},X_{0})$ = \\ \((X_{3} \land X_{2} \land \overline{X_{1}} \land \overline{X_{0}}) \lor  (X_{3} \land \overline{X_{2}} \land X_{1} \land \overline{X_{0}}) \lor (X_{3} \land \overline{X_{2}} \land \overline{X_{1}} \land X_{0}) \lor\\ (\overline{X_{3}} \land X_{2} \land X_{1} \land \overline{X_{0}}) \lor (\overline{X_{3}} \land X_{2} \land \overline{X_{1}} \land X_{0}) \lor (X_{3} \land \overline{X_{2}} \land X_{1} \land X_{0})\)
\end{document}
