\documentclass[a4paper,10pt]{scrartcl}
\usepackage[utf8]{inputenc}
\usepackage[ngerman]{babel}
\usepackage[T1]{fontenc}
\usepackage{amsmath}
\usepackage{amssymb}
\usepackage{graphicx}
\usepackage{qtree}
\input kvmacros

\begin{document}
\title{Praktikum Rechnerstrukturen 01}
\author{Jan Lukas Deichmann / Jan-Tjorve Sobieski}
\graphicspath{{images/}}
\maketitle
\noindent
\\\underline{1.2c i}\\\\
Gesucht: $x_{3} \land x_{2} \land x_{1} \land x_{0}$ (4AND)\\\\
$(x_{3} \land x_{2}) \land (x_{1} \land x_{0})$ (Assoziativität) \\
$\Leftrightarrow$ $x_{3} \land x_{2} \land (x_{1} \land x_{0})$ (Assoziativität)\\
$\Leftrightarrow$ $x_{3} \land x_{2} \land x_{1} \land x_{0}$\\\\
Gesucht: $x_{2} \land x_{1} \land x_{0}$ (3AND)\\\\
$(x_{2} \land x_{1}) \land x_{0}$ (Assoziativität) \\
$\Leftrightarrow$ $x_{2} \land x_{1} \land x_{0}$ \\\\
\underline{1.2c ii}
\begin{table}[h]
\begin{tabular}{l r}
    \Tree[.$\lor$ [.$x_{0}$ ]
    [.$\lor$ [.$x_{2}$ ]
        [.$x_{1}$ ]]]
 & \Tree[.$\lor$3 [.$x_{0}$ ]
 [.$x_{1}$ ] [.$x_{2}$ ] ]
\end{tabular}
\end{table}
\\Die Tiefe des Ausdrucks verändert sich nicht, da ein normaler Operatorbaum mit einem erweiterten Operatorbaum nicht verglichen werden kann.
\newpage
\noindent
\underline{1.2d}\\\\
Sei f: $\mathbb{B}^4$ $\rightarrow$ $\mathbb{B}^1$\\\\
$f(x_{3},x_{2},x_{1},x_{0})$ = \\\( (\lnot x_{2} \land x_{1} \land x_{0}) \lor (\lnot x_{3} \land x_{1} \land x_{0}) \lor (\lnot x_{3} \land x_{2} \land x_{0}) \lor (x_{2} \land \lnot x_{1} \land x_{0})\)\\\\\\
\underline{1.2e}\\\\
Sei $\mathbb{B}$ = \{0,1\} und D $\subseteq$ $\mathbb{B}^n$\\\\
Sei $\mathbb{B}_{4,1}$:= \{ f; f: $\mathbb{B}^4$ $\rightarrow$ $\mathbb{B}^1$\}\\\\
$f(x_{3},x_{2},x_{1},x_{0})$ = \\\( (\lnot x_{2} \land x_{1} \land x_{0}) \lor (\lnot x_{3} \land x_{1} \land x_{0}) \lor (\lnot x_{3} \land x_{2} \land x_{0}) \lor (x_{2} \land \lnot x_{1} \land x_{0}) \\\lor (\lnot x_{3} \land \lnot x_{2} \land x_{1} \land \lnot x_{0})\)\\\\
\newpage
\noindent
\underline{1.3}\\\\
Beschreibung der Funktion:\\
Ein Volladdierer, aufgebaut aus zwei Halbaddierern.\\\\
\underline{1.4a i}
\begin{tabbing}
\underline{$x_{3}$ $x_{2}$ $x_{1}$ $x_{0}$ Y} \\
$x_{3}$ \=$x_{2}$ \=$x_{1}$ \=$x_{0}$ \=Y \kill
0 \>0 \>1 \>1 \>1\\
0 \>1 \>0 \>1 \>1\\
0 \>1 \>1 \>0 \>1\\
1 \>0 \>0 \>1 \>1\\
1 \>0 \>1 \>0 \>1\\
1 \>1 \>0 \>0 \>1\\\\\\
\end{tabbing}
\underline{1.4a ii}\\\\
 $f(x_{3},x_{2},x_{1},x_{0})$ = \\ \((x_{3} \land x_{2} \land \overline{x_{1}} \land \overline{x_{0}}) \lor  (x_{3} \land \overline{x_{2}} \land x_{1} \land \overline{x_{0}}) \lor (x_{3} \land \overline{x_{2}} \land \overline{x_{1}} \land x_{0}) \lor\\ (\overline{x_{3}} \land x_{2} \land x_{1} \land \overline{x_{0}}) \lor (\overline{x_{3}} \land x_{2} \land \overline{x_{1}} \land x_{0}) \lor (x_{3} \land \overline{x_{2}} \land x_{1} \land x_{0})\)\\\\
 \karnaughmap{4}{$KV$}{{$x_{3}$}{$x_{1}$}{$x_{2}$}{$x_{0}$}}{0001 0110 0110 1000}{}
\end{document}
